\chapter{Counting}
\minitoc

This chapter is based on the materials in \it{Concrete Mathematics} written by Graham, Knuth, and Patashnik.

\section{Recurrence relations}
\subsection{The Tower of Hanoi}
This is a very basic problem in Mathematics. Suppose you are given a tower of $n$ disks, stacked in decreasing size on one of three pegs. You want to move all of the disks to another pegs with the minimum number of moves. 
It's not that hard to see that the best strategy for moving disks is
\begin{itemize}
	\item Move the first $n-1$ disks from the initial peg to a temporary peg. 
	\item Move the last disk from the initial peg to a target peg.
	\item Move $n-1$ disks from the temporary peg to the target peg.
\end{itemize}
Let $T_n$ denote the smallest number of moves you need to move $n$ disks from any peg to any other peg. It would not hard to see that the answer to this according to our paradigm is
\[
	\begin{cases}
		T_0 = 0; \\
		T_n = 2T_{n-1} + 1, &\text{for $n > 0$}.
	\end{cases}
\]
\subsection{Lines in the Plane}
Let's move to another example. Suppose you have $n$ lines, with infinite length, on the $\bb{R}^2$. You want to find the maximum number of distinct regions created by the set of $n$ lines you have. Let $L_n$ denote the maximum number of regions made by $n$ lines. It's not hard to see that $L_0 = 1$, $L_1 = 2$, and $L_2 = 4$. But, how do we generalize this?


\subsection{The Josephus Problem}
\subsection{Exercises}

\section{Matrix exponentiation}

\section{Binomial coefficient}

\section{Pigeonhole principle}

\section{Inclusion-exclusion principle}

\section{Subsets and Permutations}

\section{Indistinguishable and distinguishable objects}

\section{Lattice problems}

\section{Counting on graphs}

\section{Polya enumeration theorem}